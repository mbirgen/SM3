


Calculating means and other simple statistics is a matter of using the
right function in R.  The \pkg{mosaic} package --- which you should
load in to R as shown in Section \ref{sec:loading-mosaic} --- makes it
straightforward to calculate either a ``grand'' statistic or a
``group-wise'' statistic.   To illustrate:

\noindent Load the \pkg{mosaic} package, needed only once per session:
\begin{Schunk}
\begin{Sinput}
> require(mosaic) 
\end{Sinput}
\end{Schunk}

\noindent Read in data you are interested in analyzing, for instance
the Cherry-Blossom 2008 data described earlier: \datasetCherryBlossomEight
\begin{Schunk}
\begin{Sinput}
> runners = fetchData("Cherry-Blossom-2008.csv")
> names( runners )
\end{Sinput}
\begin{Soutput}
[1] "position" "division" "total"    "name"     "age"     
[6] "place"    "net"      "gun"      "sex"     
\end{Soutput}
\end{Schunk}

\noindent Calculate a grand mean on the ``gun'' time --- the time from
the start of the race, signalled by a gun, and when each runner
crossed the finish line:
\begin{Schunk}
\begin{Sinput}
> mean( gun, data=runners )
\end{Sinput}
\begin{Soutput}
[1] 93.7
\end{Soutput}
\end{Schunk}
\noindent Or other ``grand'' statistics.

\begin{Schunk}
\begin{Sinput}
> median( gun, data=runners )
\end{Sinput}
\begin{Soutput}
[1] 93.7
\end{Soutput}
\begin{Sinput}
> sd( gun, data=runners )
\end{Sinput}
\begin{Soutput}
[1] 15
\end{Soutput}
\end{Schunk}

To tell R that you want to break the statistics down by groups, use
the \verb+~+ notation, pronounced ``\newword{tilde}.''  You will be
using this notation frequently in building models.  It means, ``model
by'' or ``broken down by'' or ``versus.''
\begin{Schunk}
\begin{Sinput}
> mean( gun ~ sex, data=runners )
\end{Sinput}
\begin{Soutput}
  sex    S    N Missing
1   F 98.8 6397       0
2   M 88.3 5905       0
\end{Soutput}
\end{Schunk}

Other statistics work the same way, for instance,
\begin{Schunk}
\begin{Sinput}
> sd( gun ~ sex, data=runners )
\end{Sinput}
\begin{Soutput}
  sex    S    N Missing
1   F 13.3 6397       0
2   M 14.7 5905       0
\end{Soutput}
\end{Schunk}

Another example ... wage broken down by sector of the economy, using
data :
\begin{Schunk}
\begin{Sinput}
> cps = fetchData("cps.csv")
> mean( wage ~ sector, data=cps )
\end{Sinput}
\begin{Soutput}
    sector     S   N Missing
1 clerical  7.42  97       0
2    const  9.50  20       0
3    manag 12.70  55       0
4    manuf  8.04  68       0
5    other  8.50  68       0
6     prof 11.95 105       0
7    sales  7.59  38       0
8  service  6.54  83       0
\end{Soutput}
\end{Schunk}

In the Whickham smoking data example, the outcome for each person was
not a number but a category: Alive or Dead at the time of the
follow-up.  \datasetWhickham
\begin{Schunk}
\begin{Sinput}
> w = fetchData("whickham.csv")
> names(w)
\end{Sinput}
\begin{Soutput}
[1] "outcome" "smoker"  "age"    
\end{Soutput}
\begin{Sinput}
> levels(w$outcome)
\end{Sinput}
\begin{Soutput}
[1] "Alive" "Dead" 
\end{Soutput}
\end{Schunk}

To find the proportion of people who were alive at the end of the
20-year follow-up period, you can use a computational trick.  Convert
the \VN{outcome} variable to \texttt{TRUE} or \texttt{FALSE} to
indicate whether an individual is alive, then take the mean of the
true/false values.  R, like many computer languages, treats
\texttt{TRUE} as 1 and \texttt{FALSE} as 0 for the purposes of doing arithmetic.
\begin{Schunk}
\begin{Sinput}
> mean( outcome=="Alive", data=w )
\end{Sinput}
\begin{Soutput}
[1] 0.719
\end{Soutput}
\end{Schunk}

Here's the breakdown according to smoking status:
\begin{Schunk}
\begin{Sinput}
> mean( outcome=="Alive" ~ smoker, data=w )
\end{Sinput}
\begin{Soutput}
  smoker     S   N Missing
1     No 0.686 732       0
2    Yes 0.761 582       0
\end{Soutput}
\end{Schunk}

A more meaningful question is whether smokers are different from
non-smokers when holding other variables constant, such as age.  To
address this question, you need to add age into the model.

It might be natural to consider
each age --- 35, 36, 37, and so on --- as a separate group, but you
won't get very many members of each group.  And, likely, the data for
35 year-olds has quite a lot to say about 36 year-olds, so it doesn't
make sense to treat them as completely separate groups.

You can use the \function{cut} function to divide up a quantitative
variable into groups.  You get to specify the breaks between groups.
\begin{Schunk}
\begin{Sinput}
> w$ageGroups = cut(w$age, breaks=c(0,30,40,53,64,75,100) )
> mean( outcome=="Alive" ~ ageGroups, data=w )
\end{Sinput}
\begin{Soutput}
  ageGroups     S   N Missing
1    (0,30] 0.979 288       0
2   (30,40] 0.948 252       0
3   (40,53] 0.832 280       0
4   (53,64] 0.625 251       0
5   (64,75] 0.201 169       0
6  (75,100] 0.000  74       0
\end{Soutput}
\begin{Sinput}
> mean( outcome=="Alive" ~ smoker + ageGroups, data=w )
\end{Sinput}
\begin{Soutput}
   smoker ageGroups     S   N Missing
1      No    (0,30] 0.982 165       0
2     Yes    (0,30] 0.976 123       0
3      No   (30,40] 0.955 134       0
4     Yes   (30,40] 0.941 118       0
5      No   (40,53] 0.876 113       0
6     Yes   (40,53] 0.802 167       0
7      No   (53,64] 0.669 127       0
8     Yes   (53,64] 0.581 124       0
9      No   (64,75] 0.214 131       0
10    Yes   (64,75] 0.158  38       0
11     No  (75,100] 0.000  62       0
12    Yes  (75,100] 0.000  12       0
\end{Soutput}
\end{Schunk}

With modeling techniques, to be introduced in later chapters, you can use
quantitative variables without the need to divide them into groups.

\subsection{Model Values and Residuals}


A group-wise model tells you a model value for each group, but often
you will need these in a case-by-case format: the model value for each
case in a data set.  The \function{fitted} function carries out
this simple calculation, taking each case in turn, figuring out which
group it belongs to, and then returning the set of model values for
all the cases.  It requires two arguments: the group-wise model and
the data on which the model was based.  For example:

\index{P}{fitted}
\index{P}{fetchData}

\begin{Schunk}
\begin{Sinput}
> kids = fetchData("kidsfeet.csv")
> mod = mean( width ~ sex, data=kids )
> fitted( mod, kids )
\end{Sinput}
\begin{Soutput}
 [1] 9.19 9.19 9.19 9.19 9.19 9.19 9.19 8.78 8.78 9.19 9.19
[12] 9.19 9.19 9.19 8.78 8.78 8.78 8.78 8.78 8.78 9.19 9.19
[23] 8.78 8.78 8.78 9.19 8.78 9.19 9.19 9.19 8.78 8.78 8.78
[34] 9.19 9.19 8.78 8.78 8.78 8.78
\end{Soutput}
\end{Schunk}

The residuals are found by subtracting the case-by-case model value
from the actual values for each case.  
\begin{Schunk}
\begin{Sinput}
> res = kids$width - fitted(mod, kids)
\end{Sinput}
\end{Schunk}
Take care to use the same quantitative variable (\VN{width} in this case) from the
data as was used in constructing the model.

The \function{var} function will calculate the variance:
\begin{Schunk}
\begin{Sinput}
> var( kids$width )  # overall variation
\end{Sinput}
\begin{Soutput}
 var 
0.26 
\end{Soutput}
\begin{Sinput}
> var( fitted(mod, kids) ) # variation in model values
\end{Sinput}
\begin{Soutput}
   var 
0.0422 
\end{Soutput}
\begin{Sinput}
> var( kids$width - fitted(mod, kids) ) # residual variation
\end{Sinput}
\begin{Soutput}
  var 
0.217 
\end{Soutput}
\end{Schunk}

