


Again, you will need the \pkg{mosaic} package in R, which provides
some of the basic operators for constructing sampling and resampling distributions.
\begin{Schunk}
\begin{Sinput}
> require(mosaic)
\end{Sinput}
\end{Schunk}

The examples will be based on 
the Cherry-Blossom 2008 data described earlier: \datasetCherryBlossomEight
\begin{Schunk}
\begin{Sinput}
> runners = fetchData("Cherry-Blossom-2008.csv")
> names( runners )
\end{Sinput}
\begin{Soutput}
[1] "position" "division" "total"    "name"     "age"      "place"   
[7] "net"      "gun"      "sex"     
\end{Soutput}
\end{Schunk}

\subsection{Finding a Sampling Distribution through Bootstrapping}
\label{sec:find-sampling-distribution}

Your data are typically a sample from a population.  Collecting the
sample is usually hard work.  Just to illustrate the context of
a sampling distribution, here is a simulation of selecting a sample of
size $n=100$ from the population of runners.  It's essential to keep
in mind that {\bf you do not usually} pull out your sample using the
computer in this way.  Instead, you go into the field or laboratory and collect your data.
\begin{Schunk}
\begin{Sinput}
> mysamp = deal( runners, 100 ) # A Simulation of sampling
\end{Sinput}
\end{Schunk}

Now that you have a sample, you can calculate the sample statistic
that's of interest to you.  For instance:
\begin{Schunk}
\begin{Sinput}
> mean( gun ~ sex, data=mysamp )
\end{Sinput}
\begin{Soutput}
  sex     S  N Missing
1   F 100.4 47       0
2   M  88.1 53       0
\end{Soutput}
\end{Schunk}
Note that the results are slightly different from those found above
using the whole population.  That's to be expected, since the sample
is just a random part of the population.  But ordinarily, you will not
know what the population values are; all you have to work with is your
sample.  

Theoretically, the sampling distribution reflects the variation from
one randomly dealt sample to another, where each sample is taken from
the population.  In practice, your only ready access to the population
is through your sample.  So, to simulate the process of random
sampling, re-sampling is used and the re-sampling distribution is used
as a convenient approximation to the sampling distribution.

Re-sampling involves drawing from the set of cases in your sample with replacement.  To
illustrate, consider this example of a 
very small, simple set: 
the numbers 1 to 5:
\begin{Schunk}
\begin{Sinput}
> nums = c(1,2,3,4,5)
> nums
\end{Sinput}
\begin{Soutput}
[1] 1 2 3 4 5
\end{Soutput}
\end{Schunk}

Each resample of size $n$ consists of $n$ members from the set, but in
any one resample
each member might appear more than once and some might not appear at
all.  Here are three different resamples from \texttt{nums}:
\begin{Schunk}
\begin{Sinput}
> resample(nums)
\end{Sinput}
\begin{Soutput}
[1] 1 1 4 5 4
\end{Soutput}
\begin{Sinput}
> resample(nums)
\end{Sinput}
\begin{Soutput}
[1] 4 2 3 4 2
\end{Soutput}
\begin{Sinput}
> resample(nums)
\end{Sinput}
\begin{Soutput}
[1] 5 2 2 1 1
\end{Soutput}
\end{Schunk}

To use resampling to estimate the sampling distribution of a
statistic, you apply the calculation of the statistic to a resampled
version of your sample.  For instance, here are the group-wise means
from one resample of the running sample:
\begin{Schunk}
\begin{Sinput}
> mean( gun ~ sex, data=resample(mysamp) )
\end{Sinput}
\begin{Soutput}
  sex    S  N Missing
1   F 97.9 47       0
2   M 92.5 53       0
\end{Soutput}
\end{Schunk}
And here is another:
\begin{Schunk}
\begin{Sinput}
> mean( gun ~ sex, data=resample(mysamp) )
\end{Sinput}
\begin{Soutput}
  sex     S  N Missing
1   F 101.3 45       0
2   M  87.1 55       0
\end{Soutput}
\end{Schunk}

\index{C}{bootstrapping}

The bootstrap procedure involves conducting many such trials and examining the
variation from one trial to the next.  

The \function{do} function lets
you automate the collection of multiple trials.  For instance, here
are five trials carried out using \function{do}:
\begin{Schunk}
\begin{Sinput}
> do(5) * mean( gun ~ sex, data=resample(mysamp) )
\end{Sinput}
\begin{Soutput}
   sex     S  N Missing do.rep
1    F  97.8 46       0      1
2    M  89.9 54       0      1
3    F  99.2 44       0      2
4    M  87.0 56       0      2
5    F 100.0 40       0      3
6    M  90.3 60       0      3
7    F 100.6 44       0      4
8    M  87.6 56       0      4
9    F 102.8 56       0      5
10   M  88.2 44       0      5
\end{Soutput}
\end{Schunk}

Typically, you will use several hundred trials for bootstrapping.  The
most common way to summarize the variation in bootstrap trials, you
can calculate a 95\% coverage interval.  (When applied to a sampling
distribution, the coverage interval is called a \newword{confidence interval}.)

To do the computation, give a name to  the results of the repeated
bootstrap trials, here it's called \texttt{trials}:
\begin{Schunk}
\begin{Sinput}
> trials = do(500) * mean( gun ~ sex, data=resample(mysamp) )
\end{Sinput}
\end{Schunk}
\begin{Schunk}
\begin{Sinput}
> trials
\end{Sinput}
\end{Schunk}
\begin{Schunk}
\begin{Soutput}
  sex    S  N Missing do.rep
1   F 99.6 44       0      1
2   M 87.4 56       0      1
3   F 97.1 48       0      2
4   M 88.0 52       0      2
5   F 98.0 55       0      3
6   M 88.4 45       0      3
... for 1000 rows altogether ...
\end{Soutput}
\end{Schunk}

Computing the coverage intervals can be done using \function{qdata} on
the columns, or, for convenience, use the \function{confint} function.
\begin{Schunk}
\begin{Sinput}
> confint(trials)
\end{Sinput}
\begin{Soutput}
  sex 2.5% 97.5%
1   F 96.9 104.2
2   M 83.6  92.4
\end{Soutput}
\end{Schunk}

The idea of sampling distributions is based on
drawing at random from the population, not resampling.  Ordinarily,
you can't do this calculation since you don't have the population at
hand.  But in this example, we happen to have the data for all
runners.  Here's the population-based confidence interval for the mean
running time, with sample size $n=100$, broken down by sex:

\index{P}{confint}
\index{C}{confidence intervals}

\begin{Schunk}
\begin{Sinput}
> trials = do(500) * mean( gun ~ sex, data=deal(runners,100) )
> confint(trials)
\end{Sinput}
\begin{Soutput}
  sex 2.5% 97.5%
1   F 95.1 102.5
2   M 84.1  92.2
\end{Soutput}
\end{Schunk}

\index{C}{confidence level}

The output of \function{confint} shows the lower and upper limits of
the confidence interval for each of the groups.  The labels on the
columns indicate the confidence level.  By default, the interval is at
the 95\% level, and so the interval runs from the 2.5 percentile to
the 97.5 percentile. 

Historically, statisticians have been concerned with very small
samples: say $n=2$ or $n=3$.  Even in this era of huge data sets, such
small sample sizes often are encountered in laboratory experiments,
etc. Bootstrapping cannot work well with such
small samples, and other techniques are needed to simulate sampling
variability.  Many of these techniques are based in algebra and
probability theory, and give somewhat complex formulas for calculating confidence
intervals from data. The formulas are often found in textbooks, but for most of the modeling techniques you will use
in later chapters, appropriate formulas for confidence intervals have been implemented in
software.  For other modeling techniques, bootstrapping is used to 
find the confidence intervals.  But keep in mind that bootstrapping
can only be effective when the sample size $n$ is one or two dozen or
larger.  


\subsection{Computing Grade-Point Averages}

The grade-point average is a kind of group-wise mean, where the group
is an individual student.  This is not the usual way of looking at
things for a student, who sees only his or her own grades.  But
institutions have data on many students.

The data files \texttt{grades.csv} and \texttt{courses.csv} are drawn from
an institutional database at a college.  They give the grades for more
than 400 students who graduated in year 2005.  Another file,
\texttt{grade-to-number.csv}, gives the rules used by the institution
in converting letter grades to numbers.  

The data files are part of a \newword{relational data base}, a very
important way of managing large amounts of data used by private and
public institutions, corporations and governments --- it's the basis
for a multi-billion dollar segment of the economy.  Ordinarily,
relational data bases are queried using special-purpose computer
languages that sort, extract, and combine the data.  Here are the R
commands for converting the letter grades to numbers and extracting
the data for one student:
\begin{Schunk}
\begin{Sinput}
> grades = fetchData("grades.csv")
> gp = fetchData("grade-to-number.csv")
> all.students = merge(grades, gp)
> one.student = subset( all.students, sid=="S31509" )
\end{Sinput}
\end{Schunk}
\begin{Schunk}
\begin{Sinput}
> one.student
\end{Sinput}
\end{Schunk}
\begin{Schunk}
\begin{Soutput}
     grade    sid   sessionID gradepoint
192      A S31509 session3443          4
543      A S31509 session2308          4
674      A S31509 session2851          4
1280     A S31509 session2737          4
... for 13 cases altogether ...
\end{Soutput}
\end{Schunk}

Calculating the mean grade-point for the one student is a simple matter:
\begin{Schunk}
\begin{Sinput}
> mean( gradepoint, data=one.student )
\end{Sinput}
\begin{Soutput}
[1] 3.6
\end{Soutput}
\end{Schunk}

It's equally straightforward to calculate the grade-point averages
for all students as individuals:
\begin{Schunk}
\begin{Sinput}
> mean( gradepoint ~ sid, data=all.students )
\end{Sinput}
\end{Schunk}
\begin{Schunk}
\begin{Soutput}
     sid    S  N Missing
1 S31185 2.41  8       0
2 S31188 3.02 16       2
3 S31191 3.21 14       1
4 S31194 3.36 12       0
5 S31197 3.36 13       0
... for 443 cases altogether ...
\end{Soutput}
\end{Schunk}

\index{P}{do}
\index{P}{confint}
\index{P}{mean}

Bootstrapping can be used to find the confidence interval on the
grade-point average for each student:
\begin{Schunk}
\begin{Sinput}
> trials = do(100)*mean(gradepoint ~ sid, data=resample(all.students) )
\end{Sinput}
\end{Schunk}

\begin{Schunk}
\begin{Sinput}
> confint(trials)
\end{Sinput}
\end{Schunk}

\begin{Schunk}
\begin{Soutput}
     sid 2.5% 97.5%
1 S31185 1.65  3.14
2 S31188 2.56  3.47
3 S31191 3.01  3.43
4 S31194 3.05  3.70
5 S31197 3.20  3.52
... for 443 students altogether ...
\end{Soutput}
\end{Schunk}

It's important to point out that there are other methods for calculating
confidence intervals that are based on the standard deviation of the
data.  Formulas and procedures for such methods are given in just
about every standard introductory statistics book and would certainly
be used instead of bootstrapping in a simple calculation of the sort
illustrated here.  

However, such formulas don't go to the heart of the problem: 
accounting for variation in the grades and the contribution from
different sources of that variation.  For example, some of the
variation in this student's grades might be due systematically to
improvement over time or due to differences between instructor's
practices.  The modeling techniques introduced in the following
chapters provide a means to examine and quantify the different sources
of variation.
