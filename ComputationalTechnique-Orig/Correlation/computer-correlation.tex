


The coefficient of determination, $R^2$, 
compares the variation in the response variable to the variation
in the fitted model value.  It can be calculated as a ratio of variances: \datasetSwimming
\begin{Schunk}
\begin{Sinput}
> swim = fetchData("swim100m.csv")
> mod = lm(time ~ year + sex, data=swim)
> var(fitted(mod)) / var(swim$time)
\end{Sinput}
\begin{Soutput}
  var 
0.844 
\end{Soutput}
\end{Schunk}
\index{C}{coef. of determination!computing}
\index{P}{var@\texttt{var}}
\index{P}{Descriptive Statistics!var@\texttt{var}}

The convenience function \function{rsquared} does the calculation for you:
\begin{Schunk}
\begin{Sinput}
> rsquared(mod)
\end{Sinput}
\begin{Soutput}
[1] 0.844
\end{Soutput}
\end{Schunk}

The \newword{regression report} is a standard way of summarizing
models.  Such a report is produced by most statistical software
packages and used in many fields.  The first part of the table
contains the coefficients --- labeled ``Estimate'' --- along with
other information that will be introduced starting in Chapter
\ref{chap:confidence}.  The $R^2$ statistic is a standard part of the
report; look at the second line from the bottom.
\begin{Schunk}
\begin{Sinput}
> summary(mod)
\end{Sinput}
\begin{Soutput}
...
            Estimate Std. Error t value Pr(>|t|)    
(Intercept) 555.7168    33.7999   16.44  < 2e-16 ***
year         -0.2515     0.0173  -14.52  < 2e-16 ***
sexM         -9.7980     1.0129   -9.67  8.8e-14 ***
---
Signif. codes:  0 '***' 0.001 '**' 0.01 '*' 0.05 '.' 0.1 ' ' 1 

Residual standard error: 3.98 on 59 degrees of freedom
Multiple R-squared: 0.844,	Adjusted R-squared: 0.839 
F-statistic:  160 on 2 and 59 DF,  p-value: <2e-16 
\end{Soutput}
\end{Schunk}


Occasionally, you may be interested in the correlation coefficient $r$ between two
quantities.  
You can, of course, compute $r$ by fitting a model, finding $R^2$,
and taking a square root.
\begin{Schunk}
\begin{Sinput}
> mod2 = lm(time ~ year, data=swim)
> coef(mod2)
\end{Sinput}
\begin{Soutput}
(Intercept)        year 
     567.24       -0.26 
\end{Soutput}
\begin{Sinput}
> sqrt(rsquared(mod2))
\end{Sinput}
\begin{Soutput}
[1] 0.772
\end{Soutput}
\end{Schunk}

The \function{cor} function computes this directly:
\begin{Schunk}
\begin{Sinput}
> cor(swim$time, swim$year)
\end{Sinput}
\begin{Soutput}
[1] -0.772
\end{Soutput}
\end{Schunk}

Note that the negative sign on $r$ indicates that record swim \VN{time}
decreases as \VN{year} increases.  This information about the
direction of change is contained in the sign of the coefficient from
the model.  The magnitude of the coefficient tells how fast the
\VN{time} is changing (with units of seconds per year).  The
correlation coefficient (like $R^2$) is without units.

Keep in mind that the correlation coefficient $r$ summarizes only the
simple linear model \model{A}{B} where B is quantitative.  But the
coefficient of determination, $R^2$, summarizes any model; it is much
more useful.  If you want to see the direction of change, look at the
sign of the coefficient.

