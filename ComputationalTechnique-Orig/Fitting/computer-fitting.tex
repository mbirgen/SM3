


Using the \code{lm} software is largely a matter of familiarity 
with the model design language described in Chapter \ref{chap:language}.  
Computing the fitted model values and the residuals is 
done with the \code{fitted} and \code{resid}.  These operators take
a model as an input.  To illustrate:
\index{P}{lm@\texttt{lm}}
\index{P}{fitted@\texttt{fitted}}
\index{P}{resid@\texttt{resid}}
\index{P}{Modeling!lm@\texttt{lm}}
\index{P}{Modeling!fitted@\texttt{fitted}}
\index{P}{Modeling!resid@\texttt{resid}}

\index{C}{residual!computing}
\index{C}{fitted model values!computing}
\index{C}{fitting!software}
\index{C}{regression report!computing}



\begin{Schunk}
\begin{Sinput}
> swim = fetchData("swim100m.csv")
> mod1 = lm(time ~ year + sex, data=swim)
> coef(mod1)
\end{Sinput}
\begin{Soutput}
(Intercept)        year        sexM 
    555.717      -0.251      -9.798 
\end{Soutput}
\end{Schunk}

\datasetSwimming

Once you have constucted the model, you can use \code{fitted} and
\code{resid}: 
\begin{Schunk}
\begin{Sinput}
> fitted(mod1)
\end{Sinput}
\end{Schunk}
\begin{Schunk}
\begin{Soutput}
   1    2    3    4    5    6    7    8    9   10   11   12 
66.9 66.1 65.6 65.1 63.6 63.1 62.6 62.1 59.6 59.3 59.1 57.1 
... for 62 cases altogether ...
\end{Soutput}
\end{Schunk}

\subsection{Sums of Squares}

\index{C}{sum of squares!computing}

Computations can be performed on the fitted model values and the
residuals, just like any other quantity:
\begin{Schunk}
\begin{Sinput}
> mean(fitted(mod1))
\end{Sinput}
\begin{Soutput}
mean 
59.9 
\end{Soutput}
\begin{Sinput}
> var(resid(mod1))
\end{Sinput}
\begin{Soutput}
 var 
15.3 
\end{Soutput}
\begin{Sinput}
> sd(resid(mod1))
\end{Sinput}
\begin{Soutput}
  sd 
3.92 
\end{Soutput}
\begin{Sinput}
> summary(resid(mod1))
\end{Sinput}
\begin{Soutput}
   Min. 1st Qu.  Median    Mean 3rd Qu.    Max. 
  -4.70   -2.70   -0.60    0.00    1.28   19.10 
\end{Soutput}
\end{Schunk}

Sums of squares are very important in statistics.  Here's how to
calculate them for the response values, the fitted model values, and
the residuals:
\begin{Schunk}
\begin{Sinput}
> sum(swim$time^2)
\end{Sinput}
\begin{Soutput}
[1] 228635
\end{Soutput}
\begin{Sinput}
> sum(fitted(mod1)^2)
\end{Sinput}
\begin{Soutput}
[1] 227699
\end{Soutput}
\begin{Sinput}
> sum(resid(mod1)^2)
\end{Sinput}
\begin{Soutput}
[1] 936
\end{Soutput}
\end{Schunk}
The partitioning of variation by models is seen by the way 
\index{C}{partitioning!sums of squares}
\index{C}{sum of squares!partitioning}
the sum of squares of the fitted and the residuals add up to the sum of squares of the response:
\begin{Schunk}
\begin{Sinput}
> 227699 + 935.8
\end{Sinput}
\begin{Soutput}
[1] 228635
\end{Soutput}
\end{Schunk}

Don't forget the squaring stage of the operation!  The sum of the
residuals (without squaring) 
is very different from the sum of squares of the residuals:
\begin{Schunk}
\begin{Sinput}
> sum(resid(mod1))
\end{Sinput}
\begin{Soutput}
[1] 1.85e-14
\end{Soutput}
\begin{Sinput}
> sum(resid(mod1)^2)
\end{Sinput}
\begin{Soutput}
[1] 936
\end{Soutput}
\end{Schunk}
Take care in reading numbers formatted like 
\code{1.849e-14}.   The notation stands for $1.849 \times 10^{-14}$.
That number, $0.00000000000001849$, is effectively zero compared to the
residuals themselves!
\index{C}{scientific notation}
\index{C}{zero}


\subsection{Redundancy}

\index{C}{redundancy!and NA}
\index{C}{coefficients!NA (redundancy)}
The \code{lm} operator will automatically detect redundancy and deal
with it by leaving the redundant terms out of the model.  

To see how redundancy is handled, here is an example with a
constructed redundant variable in the swimming dataset.  
The following statement adds a new variable to the
dataframe counting how many years after the end of World War II each
record was established:
\begin{Schunk}
\begin{Sinput}
> swim$afterwar = swim$year - 1945
\end{Sinput}
\end{Schunk}

Here is a model that doesn't involve redundancy
\begin{Schunk}
\begin{Sinput}
> mod1 = lm( time ~ year + sex, data=swim)
> coef(mod1)
\end{Sinput}
\begin{Soutput}
(Intercept)        year        sexM 
    555.717      -0.251      -9.798 
\end{Soutput}
\end{Schunk}

When the redundant variable is added in, \code{lm} successfully
detects the redundancy and handles it.  This is indicated by a
coefficient of NA on the redundant variable.
\begin{Schunk}
\begin{Sinput}
> mod2 = lm( time ~ year + sex + afterwar, data=swim)
> coef(mod2)
\end{Sinput}
\begin{Soutput}
(Intercept)        year        sexM    afterwar 
    555.717      -0.251      -9.798          NA 
\end{Soutput}
\end{Schunk}

In the absence of redundancy, the model coefficients don't depend on
the order in which the model terms are specified.  But this is not the
case when there is redundancy, since any redundancy is blamed on the
later variables.  For instance, here \VN{afterwar} has been put first
in the explanatory terms, so \code{lm} identifies \VN{year} as the
redundant variable:
\begin{Schunk}
\begin{Sinput}
> mod3 = lm( time ~ afterwar + year + sex, data=swim)
> coef(mod3)
\end{Sinput}
\begin{Soutput}
(Intercept)    afterwar        year        sexM 
     66.620      -0.251          NA      -9.798 
\end{Soutput}
\end{Schunk}

\index{C}{redundancy!fitted model values}
Even though the coefficients are different, the fitted model values
and the residuals are exactly the same (to within computer round-off)
regardless of the order of the
model terms.
\begin{Schunk}
\begin{Sinput}
> fitted(mod2)
\end{Sinput}
\end{Schunk}
\begin{Schunk}
\begin{Soutput}
   1    2    3    4    5    6    7    8    9   10   11   12 
66.9 66.1 65.6 65.1 63.6 63.1 62.6 62.1 59.6 59.3 59.1 57.1 
... for 62 cases altogether ...
\end{Soutput}
\end{Schunk}
\begin{Schunk}
\begin{Sinput}
> fitted(mod3)
\end{Sinput}
\end{Schunk}
\begin{Schunk}
\begin{Soutput}
   1    2    3    4    5    6    7    8    9   10   11   12 
66.9 66.1 65.6 65.1 63.6 63.1 62.6 62.1 59.6 59.3 59.1 57.1 
... for 62 cases altogether ...
\end{Soutput}
\end{Schunk}

\index{C}{redundancy!in categorical variables}
Note that whenever you use a categorical variable and an intercept
term in a model, there is a redundancy.  This is not shown explicitly.
For example, here is a model with no intercept term, and both levels
of the categorical variable \VN{sex} show up with coefficients:
\begin{Schunk}
\begin{Sinput}
> lm(time ~ sex - 1, data=swim)
\end{Sinput}
\begin{Soutput}
...
sexF  sexM  
65.2  54.7  
\end{Soutput}
\end{Schunk}
If the intercept term is included (as it is by default unless
\code{-1} is used in the model formula), one of the levels is simply
dropped in the report:
\begin{Schunk}
\begin{Sinput}
> lm(time ~ sex, data=swim)
\end{Sinput}
\begin{Soutput}
...
(Intercept)         sexM  
       65.2        -10.5  
\end{Soutput}
\end{Schunk}
Remember that this coefficient report implicitly involves a
redundancy.  If the software had been designed differently, the report
might look like this:
\begin{Schunk}
\begin{Soutput}
(Intercept)     sexF      sexM     
       65.2       NA     -10.5  
\end{Soutput}
\end{Schunk}

