%% Author: Daniel Kaplan
%% Subject: Computing (basic data manipulation)

Read in the file \code{kidsfeet.csv}.  For each of the following, hand
in the R statements to find what is asked for.  Provide both the
command you give and the output of the command.

\begin{itemize}

\item[1] The names of the variables.\TextEntry[itemname=names]

\item[2] The mean of the foot \VN{width} variable.\TextEntry[itemname=width]

\item[3] Which of the cases are girls.\TextEntry[itemname=girls]

\item[4] The mean foot \VN{width} for the subset of data for the
  girls.\TextEntry[itemname=girlfeet]

\end{itemize}

Of no particular statistical value, but to review the use of logical
operators: 

\begin{itemize}

\item[5] The mean foot \VN{width} for the subset of data for people
whose bigger foot is left and dominant hand is also left.\TextEntry

\end{itemize}

And, for some extra practice in using logical operators:

\begin{itemize}
\item[6] The mean foot \VN{width} for the subset of data for people
who are either male or whose bigger foot matches the dominant hand. 
\TextEntry

\item[7] The mean foot \VN{width} for the subset of data for people
whose bigger foot does NOT match the dominant hand.\TextEntry

\end{itemize}

