%% Author: Daniel Kaplan
%% Subject: R (basics)


The operator \texttt{seq} generates sequences.  
Use \texttt{seq} to make the following sequences:

\begin{enumerate}
\item the integers from 1 to 10 \TextEntry

\begin{AnswerText}
Making such sequences is not in itself an important operation in
statistics, but it is an easy way to learn how to construct commands.
\begin{verbatim}
> seq(1,6)
[1] 1 2 3 4 5 6
\end{verbatim}
\end{AnswerText}

\item the integers from 5 to 15 \TextEntry

\begin{AnswerText}
\begin{verbatim}
> seq(5,15)
 [1]  5  6  7  8  9 10 11 12 13 14 15
\end{verbatim}
\end{AnswerText}


\item the integers from 1 to 10, skipping the even ones \TextEntry

\begin{AnswerText}
Skipping the even ones can be done by jumping by steps of 2.
\begin{verbatim}
> seq(1,10, by=2)
[1] 1 3 5 7 9
\end{verbatim}
Notice that 10 isn't part of the returned value, even though 10 was
specified as the end of the sequence.  When the software made the jump
from 9, it could see that the next step was outside of the specified
range and therefore stopped.
\end{AnswerText}

\item 10 evenly spaced numbers between 0 and 1 \TextEntry

\begin{AnswerText}
Of course you could figure out what the distance would be between
evenly spaced numbers in the interval 0 to 1, and use that as the
value of the \code{by=} argument. (Note that the 10 numbers will have
9 intervals between them, so the spacing is $0.11111 \ldots$.) But \code{seq} will do this for you.
\begin{verbatim}
> seq(0,1,length=10)
 [1] 0.0000000 0.1111111 0.2222222 0.3333333 0.4444444 0.5555556
 [7] 0.6666667 0.7777778 0.8888889 1.0000000
\end{verbatim}
\end{AnswerText}

\end{enumerate}

