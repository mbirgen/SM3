There are two kinds of questions that are often asked relating to
percentiles:
\begin{itemize}

\item What is the value that falls at a given percentage?  For
  instance, in the \texttt{ten-mile-race.csv} running data, how fast are the
  fastest 10\% of runners?    In R, you would ask in this way:
\begin{verbatim}
> run = fetchData("ten-mile-race.csv")
> qdata(0.10, run$net)
 10% 
4409 
\end{verbatim}
The answers is in the units of the  variable, in this case seconds.
So 10\% of the runners have net times faster than or equal to 4409 seconds.

\item What percentage falls at a given value?  For instance, what
  fraction of runners are faster than 4000 seconds?

\begin{verbatim}
> pdata(4000, run$net)
[1] 0.04029643
\end{verbatim}
The answer includes those whose \VN{net} time is exactly {\bf equal to} or
{\bf less than} 4000 seconds.

\end{itemize}

It's important to pay attention to the \texttt{p} and \texttt{q} in
the statement.  \texttt{pdata} and \texttt{qdata} ask related but
different questions.

\bigskip

\bigskip

Use \texttt{pdata} and \texttt{qdata} to answer the following
questions about the running data.

\begin{enumerate}
\item Below (or equal to)  what age are the youngest 35\% of
  runners?

\begin{itemize}
\item Which statement will do the correct calculation?
\begin{MultipleChoice}
\wrong{\code{pdata(0.35,run$age)}}
\correct{\code{qdata(0.35, run$age)}}
\wrong{\code{pdata(35,run$age)}}
\wrong{\code{qdata(35, run$age)}}
\end{MultipleChoice}

\item What will the answer be? \\
\SelectSetHoriz{31}{28,29,30,31,32,33,34,35}
\end{itemize}

\bigskip

\item What's the \VN{net} time that divides the slowest 20\% of
  runners from the rest of the runners?

\begin{itemize}
\item Which statement will do the correct calculation?
\begin{MultipleChoice}
\wrong{\code{pdata(0.20,run$net)}}
\wrong{\code{qdata(0.20, run$net)}}
\wrong{\code{pdata(0.80,run$net)}}
\correct{\code{qdata(0.80, run$net)}}
\end{MultipleChoice}

\item What will the answer be?\\
  \SelectSetHoriz{6346}{4921,5318,5988,6346,7123,7431}
seconds
\end{itemize}

\bigskip

\item What is the 95\% coverage interval on age?
\begin{itemize}
\item Which statement will do the correct calculation?
\begin{MultipleChoice}
\wrong{\code{pdata(c(0.025,0.975),run$age)}}
\correct{\code{qdata(c(0.025,0.975),run$age)}}
\wrong{\code{pdata(c(0.050,0.950),run$age)}}
\wrong{\code{qdata(c(0.050,0.950),run$age)}}
\end{MultipleChoice}

\item What will the answer be?
\begin{MultipleChoice}
\correct{22 to 60}
\wrong{20 to 65}
\wrong{25 to 59}
\wrong{20 to 60}
\end{MultipleChoice}
\end{itemize}

\bigskip

\item What fraction of runners are 30 or younger?\\
\begin{itemize}
\item Which statement will do the correct calculation?
\begin{MultipleChoice}
\correct{\code{pdata(30, run$age)}}
\wrong{\code{qdata(30, run$age)}}
\wrong{\code{pdata(30.1, run$age)}}
\wrong{\code{qdata(30.1, run$age)}}

\end{MultipleChoice}

\item What will the answer be?  \\
In percent: \SelectSetHoriz{33.7}{29.3,30.1,33.7,35.9,38.0,39.3} 
\end{itemize}

\bigskip

\item What fraction of runners are 65 or older?  (Caution: This isn't
  yet in the form of a BELOW question.)
\begin{itemize}
\item Which statement will do the correct calculation?
\begin{MultipleChoice}
\wrong{\code{pdata(65, run$age)}}
\wrong{\code{pdata(64.99, run$age)}}
\wrong{\code{pdata(65.01, run$age)}}
\wrong{\code{1-pdata(65, run$age)}}
\correct{\code{1-pdata(64.99, run$age)}}
\wrong{\code{1-pdata(65.01, run$age)}}
\end{MultipleChoice}

\begin{AnswerText}
Since \code{pdata} answers a {\bf below} question, the first step is
to ask a relevant question in that form.  You want the runners who are
65 or older, which is related to the fraction who are younger than
65.  So, \code{pdata(64.99, run$age)}  %$ 
which works because the ages
are in integer years.  Subtract that answer from 1 to get the final answer.


\end{AnswerText}


\item What will the answer be?\\
In percent: \SelectSetHoriz{1.1}{0.5,1.1,1.7,2.3,2.9} 
\end{itemize}

\item The time it takes for a runner to get to the start line after
  the starting gun is fired is
  the difference between the \VN{time} and \VN{net}. 
\begin{verbatim}
run$to.start = run$time - run$net
\end{verbatim}

\bigskip

\begin{itemize}
\item How long is it before 75\% of runners get to the start line?\\
In seconds: \SelectSetHoriz{351}{164,192,213,294,324,351} 

\bigskip


\item What fraction of runners get to the start line before one
  minute?  (Caution: the times are measured in seconds.)\\
In percent: \SelectSetHoriz{25}{10,15,19,22,25,31,34} 


\end{itemize}

\bigskip

\item What is the 95\% coverage interval on the ages of female
  runners?
\begin{MultipleChoice}
\wrong{19 to 61 years}
\wrong{22 to 61 years}
\wrong{19 to 56 years}
\correct{22 to 56 years}
\end{MultipleChoice}

\bigskip

\item What fraction of runners have a \VN{net} time BELOW 4000
  seconds?  (That is, don't include those who are at exactly 4000
  seconds.)\\
In percent: \SelectSetHoriz{4.00}{3.72,4.00,4.03,4.07,5.21} 


\end{enumerate}

