%% Author: Daniel Kaplan
%% Subject: Descriptive statisics, averages are not everything


A seemingly straightforward statistic to describe the health of a population
is \newword{average age at death}.  In 1842, the {\em Report on the
Sanitary Conditions of the Labouring Population of Great Britain}
gave these averages: ``gentlemen and persons engaged in the
professions, 45 years; tradesmen and their families, 26 years;
mechanics, servants and laborers, and their families, 16 years.''

A student questioned the accuracy of the 1842 report with this
observation: ``The mechanics, servants and laborer population wouldn't
be able to renew itself with an average age at death of 16 years.
Mothers would be dying so early in life that they couldn't possibly
raise their kids.''

Explain how an average age of death of 16 years could be quite
consistent with a ``normal'' family structure in which parents raise
their children through the child's adolescence in the teenage years.  
What other
information about ages at death would give a more complete picture of
the situation? \TextEntry