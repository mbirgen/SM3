%% Author: Daniel Kaplan
%% Subject: Describing variability (percentiles, quintiles)


Data on the distribution of economic variables, such as income, is often presented in quintiles: divisions of the group into five equal-sized parts.  

Here is a table from the US Census Bureau (Historical Income Tables from March 21, 2002) giving the distribution of income across US households in year 2000.

\bigskip

\centerline{\begin{tabular}{lrr}
 & Upper & Mean\\
Quintile & Boundary & Value\\\hline
Lowest & \$17,955 & \$10,190\\
Second & \$33,006 & \$25,334\\
Third  & \$52,272 & \$42,361\\
Fourth & \$81,960 & \$65,729\\
Fifth & --- & \$141,260\\
\end{tabular}}

\bigskip

Based on this table, calculate:
\begin{enumerate}[(a)]
\item The 20th percentile of family income. 

\SelectSetHoriz{17955}{10190,17955,33006,25334,52272,42361,81960,141260}


\item The 80th percentile of family income. 

\SelectSetHoriz{81960}{10190,17955,33006,25334,52272,42361,81960,141260}

\item The table doesn't specify the median family income but you can
  make a reasonable estimate of it.  Pick the closest one.

\SelectSetHoriz{42500}{10000,18000,25500,42500,53000,65700}


\item Note that there is no upper boundary reported for the fifth
  quintile, and no lower boundary reported for the first quintile.
  Why? \TextEntry[itemname=boundaries]

\item From this table, what evidence is there that family income has a
  skew rather than ``normal'' distribution? \TextEntry[itemname=skew]

\end{enumerate}

