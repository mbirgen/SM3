%% Author: Daniel Kaplan
%% Subject: Describing Variability (percentiles and IQR, outliers)

Here is a small table of percentiles of typical daily calorie consumption 
of college students.

\bigskip
\centerline{\begin{tabular}{rr}
Percentile & Calories\\\hline
0 & 1400\\
5 & 1800\\
10 & 2000\\
25 & 2400\\
50 & 2600\\
75 &  2900\\
90 &  3100\\
95 &  3300\\
100 & 3700\\
\end{tabular}}
\bigskip


\begin{enumerate}[(a)]

\item What is the 50\%-coverage interval?
\begin{description}
\item[Lower Boundary] \SelectSetHoriz{2400}{1800,1900,2000,2200,2400,2500,2600}
\item[Upper Boundary] \SelectSetHoriz{2900}{2600,2750,2900,3000,3100,3200,3500}
\end{description}



\item What percentage of cases lie between 2900 and 3300? \\
\SelectSetHoriz{20}{10,20,25,30,40,50,60,70,80,90,95}


\item What is the percentile
that marks the upper end of the
95\%-coverage interval? 
\SelectSetHoriz{97.5}{75,90,92.5,95,97.5,100}

Estimate the corresponding calorie value from the table.\\
\SelectSetHoriz{3500}{2900,3000,3100,3300,3500,3700}


\item Using the 1.5 IQR rule-of-thumb for identifying an outlier,
what would be the threshold for identifying a low calorie consumption as 
an outlier?\\
\SelectSetHoriz{1650}{1450,1500,1650,1750,1800,2000}



\end{enumerate}

