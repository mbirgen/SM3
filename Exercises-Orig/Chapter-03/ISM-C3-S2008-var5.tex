%% Author: Daniel Kaplan
%% Subject: Data manipulation (taking subsets)


To exercise your ability to extract subsets of data, 
take each of the following subsets of the swimming records in 
\datasetSwimming
 and calculate
the mean and minimum swimming time for the subset.  (Answers have been
rounded to one decimal place.)
\begin{itemize}
\item All records between 1920 and 1940 (including 1920 and 1940).

Mean: \SelectSetHoriz{64.7}{54.6,60.2,64.7,69.6,71.3}

Min: \SelectSetHoriz{56.4}{56.4,60.2,64.7,69.6,71.3}

\item Female records in the 1970s and 1980s

Mean: \SelectSetHoriz{56.2}{54.1,54.7,56.2,60.2,64.7,69.6,71.3}

Min: \SelectSetHoriz{54.7}{54.1,54.7,56.2,60.2,64.7,69.6,71.3}


\item All records that are {\bf slower} than 60 seconds.  (Hint: Think
  what ``slower'' means in terms of the swimming times.)

Mean: \SelectSetHoriz{69.6}{56.2,60.2,64.7,69.6,71.3,73.2,75.8}

Min: \SelectSetHoriz{60.2}{56.2,60.2,61.5,64.7,69.6,71.3}
\end{itemize}

\begin{AnswerText}
\begin{verbatim}
> s = subset( swim, year >=1920 & year <=1940)
> mean(s$time)
[1] 64.675
> min(s$time)
[1] 56.4
> s = subset( swim, sex == 'F' & year >=1970 & year < 1990)
> mean(s$time)
[1] 56.22571
> min(s$time)
[1] 54.73
> s = subset( swim, time > 60)
> mean(s$time)
[1] 69.61667
> min(s$time)
[1] 60.2
\end{verbatim}
\end{AnswerText}

